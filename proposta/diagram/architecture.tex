\begin{tikzpicture}[
    % Estilos gerais
    etapa/.style={
        rectangle,
        draw,
        thick,
        rounded corners=2pt,
        minimum width=3.5cm,
        minimum height=1cm,
        text centered,
        font=\small
    },
    subetapa/.style={
        rectangle,
        draw,
        dashed,
        rounded corners=2pt,
        minimum width=3cm,
        minimum height=0.7cm,
        text centered,
        font=\scriptsize
    },
    arr/.style={-Latex, thick},
    node distance=1.5cm and 1cm
]

% ================================
% Etapa 1 – Revisão Bibliográfica
% ================================
\node[etapa] (et1) {
  \textbf{Etapa 1} \\ Revisão Bibliográfica
};

% ================================
% Etapa 2 – Pré‐processamento
% ================================
\node[etapa, right=of et1] (et2) {
  \textbf{Etapa 2}\\ Pré‐processamento de imagens
};

% Detalhes da Etapa 2 (dois sub‐blocos)
\node[subetapa, below=0.5cm of et2] (et2a) {
  Ajuste de tamanho, brilho, contraste 
};
\node[subetapa, below=of et2a] (et2b) {
  Organização do dataset em folds
};

% ================================
% Etapa 3 – Treinamento e Geração
% ================================
\node[etapa, right=of et2] (et3) {
  \textbf{Etapa 3}\\
  Treinamento de GAN\\
  Geração de imagens sintéticas
};

% ================================
% Etapa 4 – Treinamento NeRF
% ================================
\node[etapa, right=of et3] (et4) {
  \textbf{Etapa 4}\\
  Treinamento do modelo NeRF\\
  Criação de volumes 3D
};

% ================================
% Etapa 5 – Validação dos Resultados
% ================================
\node[etapa, right=of et4] (et5) {
  \textbf{Etapa 5}\\
  Validação dos resultados
};

% Detalhes da Etapa 5 (duas sub‐validações)
\node[subetapa, below=0.5cm of et5] (et5a) {
  Validação 2D: PSNR, SSIM, LPIPS
};
\node[subetapa, below=of et5a] (et5b) {
  Validação 3D: IoU volumétrico
};

% ================================
% Conectando as Etapas com Setas
% ================================
\draw[arr] (et1.east) -- (et2.west);
\draw[arr] (et2.east) -- (et3.west);
\draw[arr] (et3.east) -- (et4.west);
\draw[arr] (et4.east) -- (et5.west);

% ================================
% Linhas ligando sub‐etapas
% ================================
\draw[arr, dashed] (et2.south) -- (et2a.north);
\draw[arr, dashed] (et2a.south) -- (et2b.north);

\draw[arr, dashed] (et5.south) -- (et5a.north);
\draw[arr, dashed] (et5a.south) -- (et5b.north);

\end{tikzpicture}
