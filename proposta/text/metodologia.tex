\section{Metodologia}

A pesquisa será conduzida em cinco etapas principais, dispostas a seguir.


\subsection{Etapa 1 - Revisão Bibliográfica}

Esta etapa permite o levantamento bibliográfico necessário para manter o projeto atualizado, proporcionar uma fundamentação teórica sólida e subsidiar a exploração proposta.


\subsection{Etapa 2 - Pré-processamento \textit{dataset} de imanges histológicas}

  Dado um conjunto de imagens fornecido pelo % TODO adicionar referencia ao dataset
  , observa-se 2 desafios principais: a necessidade de pré-processamento das imagens e a organização do conjunto de dados em segmentos distintos. 
  
  Os \textit{NeRF}'s requerem os dados sobre a pose da câmera para cada imagem, ou seja, a posição e orientação da câmera no momento da captura da imagem. Para tal, é necessário que as imagens sejam organizadas em grupos, os quais contenham, individualmente, imagens de um único tecido e pose. Consonantemente, o \textbf{pré-processamento} das imagens se torna essencial para garantir-se o formato adequado para o treinamento do modelo, em específico, será utilizado o \textit{Nerfstudio} para facilitar o \textit{parsing} dos dados de entrada. Isso inclui técnicas como a normalização das imagens, a remoção de ruídos e artefatos e a padronização do tamanho e resolução das imagens, nos casos onde se faça necessário.

  Além disso, para efeito de treinamento do modelo, é necessário que sejam estabelecidos subconjuntos de imagens, cada qual usado em diferentes contextos. Serão criados 3 conjuntos principais: \textbf{Treinamento}, \textbf{Validação} e  \textbf{Teste}, sendo, em específico, o conjunto de treinamento, novamente, subdividido. Pretende-se que, ao fazê-lo, os resultados percam possíveis viéses e, ademais, seja possível determinar os resultados de forma mais precisa.

\subsection{Etapa 3 - Treinamento e Geração de imanges}

Nesta etapa, um modelo GAN será treinado para gerar novas imagens histológicas a partir do conjunto de dados pré-processado. O modelo GAN deverá aprender as características dos tecidos histológicos presentes no \textit{dataset}, permitindo, assim, a geração de imagens que preservem as propriedades visuais e estruturais dos tecidos.

Em específico, serão investigados os modelos DCGAN, RAGAN e WGAN-GP, evidenciados no trabalho do Botazzo \cite{rozendo2024histdataaug}. Estas GAN's assimilam o problema da não competição gerada pelo discriminador, que usualmente tem vantagem acerca do gerador, cada um da sua maneira:
%For instance, in DCGAN, the first GAN proposal with convolutional layers, the authors used batch normalization and leaky ReLU activations between D’s intermediate layers to make it more stable [ 11 ]. The loss function, however, was the Jensen–Shannon divergence [ 5], which can lead to mode collapse and vanishing gradients [ 8, 11]. The WGAN-GP [12] addresses these problems by using the Wasserstein distance as a loss function, which provides a more meaningful measure of the difference between the probability distributions. WGAN-GP also introduces a gradient penalty term that penalizes the norm of D’s gradients, encouraging more diverse generated samples by enforcing a more uniform distribution of gradients throughout the data space. The RAGAN [13] introduces the relativistic discriminator that estimates the probability that an authentic sample is more realistic than a fake sample and vice versa. It considers the relative realness of real and fake samples, providing more informative feedback to G. RAGAN delivers a more nuanced signal to G, allowing it to understand better how to generate plausible samples on its own and the actual data distribution.
\begin{itemize}
  \item \textbf{DCGAN}: 
  \item \textbf{RAGAN}:
  \item \textbf{WGAN-GP}:
\end{itemize}

As imagens geradas por esta etapa serão subsequentemente 


\subsection{Etapa 4 - Treinamento do modelo NeRF e Criação dos volumes 3D}

O treinamento do modelo NeRF será realizado utilizando as imagens geradas pelo modelo GAN e, paralelamente, com subconjuntos do grupo de treinamento, supracitado. Desta forma, poder-se-á compreender a relação entre as imagens 2D e a reconstrução 3D dos tecidos histológicos. A partir do modelo NeRF treinado, serão obtidos volumes 3D dos tecidos histológicos, permitindo a visualização das estruturas internas dos tecidos com alta fidelidade. Esses volumes serão futuramente avaliados quanto à sua qualidade e fidelidade em relação às imagens originais do \textit{dataset} na etapa de validação. 

\subsection{Etapa 5 - Validação dos resultados}

Para obter-se resultados precisos, factíveis e replicáveis, este projeto buscará utilizar-se de métricas de validação dos modelos 3D reconstruídos, seja pela comparação com modelos de referência ou pela análise de poses específicas, ou seja, imagens 2D tiradas sobre o modelo.

\begin{itemize}

  \item \textbf{Validação de imangens 2D}: Serão empregados, no campo das imagens bidimensionais, tanto o PSNR (\textit{Peak Signal-To-Noise Ratio}) quanto o SSIM (\textit{Structural Similarity Index}). O PSNR faz uma comparação direta entre 2 imagens e consegue descrever o ruído discrepante entre estas, utilizando-se de uma razão sinal-ruído em escala logarítmica. Já o SSIM trabalha com a estrutura da imagem em si, ou seja, nos agrupamentos dos pixels da referência quando comparados à imagem gerada pela arquitetura GAN, capturando melhor a percepção humana de qualidade visual do que o PSNR.

  \item \textbf{Validação de modelos 3D}: Para o domínio das reconstruções 3D, utilizar-se-á o Volumetric Intersection over Union (IoU Volumétrico). Esta métrica calcula a razão entre o volume da interseção e o volume da união do modelo reconstruído e do modelo de referência, fornecendo uma medida direta da sobreposição espacial dos volumes.

Essas métricas fornecerão uma base quantitativa robusta para avaliar e comparar a qualidade dos modelos testados, dimensionando uma alta fidelidade ao obter-se valores elevados de PSNR e SSIM e de IoU Volumétrico.

\end{itemize}

\begin{tikzpicture}[
    % Estilos gerais
    etapa/.style={
        rectangle,
        draw,
        thick,
        rounded corners=2pt,
        minimum width=3.5cm,
        minimum height=1cm,
        text centered,
        font=\small
    },
    subetapa/.style={
        rectangle,
        draw,
        dashed,
        rounded corners=2pt,
        minimum width=3cm,
        minimum height=0.7cm,
        text centered,
        font=\scriptsize
    },
    arr/.style={-Latex, thick},
    node distance=1.5cm and 1cm
]

% ================================
% Etapa 1 – Revisão Bibliográfica
% ================================
\node[etapa] (et1) {
  \textbf{Etapa 1} \\ Revisão Bibliográfica
};

% ================================
% Etapa 2 – Pré‐processamento
% ================================
\node[etapa, right=of et1] (et2) {
  \textbf{Etapa 2}\\ Pré‐processamento de imagens
};

% Detalhes da Etapa 2 (dois sub‐blocos)
\node[subetapa, below=0.5cm of et2] (et2a) {
  Ajuste de tamanho, brilho, contraste 
};
\node[subetapa, below=of et2a] (et2b) {
  Organização do dataset em folds
};

% ================================
% Etapa 3 – Treinamento e Geração
% ================================
\node[etapa, right=of et2] (et3) {
  \textbf{Etapa 3}\\
  Treinamento de GAN\\
  Geração de imagens sintéticas
};

% ================================
% Etapa 4 – Treinamento NeRF
% ================================
\node[etapa, right=of et3] (et4) {
  \textbf{Etapa 4}\\
  Treinamento do modelo NeRF\\
  Criação de volumes 3D
};

% ================================
% Etapa 5 – Validação dos Resultados
% ================================
\node[etapa, right=of et4] (et5) {
  \textbf{Etapa 5}\\
  Validação dos resultados
};

% Detalhes da Etapa 5 (duas sub‐validações)
\node[subetapa, below=0.5cm of et5] (et5a) {
  Validação 2D: PSNR, SSIM, LPIPS
};
\node[subetapa, below=of et5a] (et5b) {
  Validação 3D: IoU volumétrico
};

% ================================
% Conectando as Etapas com Setas
% ================================
\draw[arr] (et1.east) -- (et2.west);
\draw[arr] (et2.east) -- (et3.west);
\draw[arr] (et3.east) -- (et4.west);
\draw[arr] (et4.east) -- (et5.west);

% ================================
% Linhas ligando sub‐etapas
% ================================
\draw[arr, dashed] (et2.south) -- (et2a.north);
\draw[arr, dashed] (et2a.south) -- (et2b.north);

\draw[arr, dashed] (et5.south) -- (et5a.north);
\draw[arr, dashed] (et5a.south) -- (et5b.north);

\end{tikzpicture}

