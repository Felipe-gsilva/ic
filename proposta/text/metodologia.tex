\section{Metodologia}

validacao dos resultados se dara por:

\begin{itemize}
  \item processar input (adaptar dataset para input padrao nerf)
  \item data augmentation
  \item GAN's
  \item codigo nerf padrao (processar imagens para 3d)
  \item verificar metricas de validacao de modelos 3D e imagens
  \item comparacao ground truth x metodos
\end{itemize}


A pesquisa proposta será desenvolvida em etapas, descritas a seguir:

\subsection{Etapa 1 - Revisão Bibliográfica}

Esta etapa permite o levantamento bibliográfico necessário para manter o projeto atualizado, proporcionar uma fundamentação teórica sólida e subsidiar a exploração proposta.

%\subsection{Etapa 2 - Exploração de \textit{BERT}}
%
%Esta etapa é direcionada para explorar a arquitetura BERT em razão dos resultados conquistados na área de processamento de linguagem natural, especialmente a partir de variações voltadas para a cybersegurança \cite{ranade2021cybert}. O modelo que será analisado e investigado foi descrito por Seyyar et al. \cite{seyyar2022attack}, em razão dos relevantes resultados conquistados para a detecção de \textit{CyberAtaques} através de pedidos HTTP/HTTPS. O modelo foi fundamentado em BERT para a criação dos vetores de palavras e MLP para a classificação. Buscando viabilizar a aplicação do modelo BERT em questão, serão examinadas amostras de ataques dos \textit{distributed denial of service} (DDoS), \textit{man in the middle} (MitM), SQL injection e \textit{cross site scripting} (XSS) \cite{luxemburk2021detection}. A ideia é analisar estes contextos a fim de verificar a viabilidade do uso de abordagens fundamentadas em sentenças regulares e mecanismos explorados via BERT. Ao comparar a entrada com as amostras maliciosas ou não presentes nos \textit{datasets} utilizados, a etapa de MLP retornará a classificação do \textit{input} fornecido, indicando assim, se alguma ameaça está ou não presente na amostra, e qual o tipo de ataque. Outras estratégias podem ser exploradas nesta etapa a fim de aprimorar e/ou atualizar a proposta em observação as possíveis tendências observadas na literatura especializada.  
%
%\subsection{Etapa 3 - Exploração de modelo \textit{ViT}}
%    
%Modelos ViT representam uma variação de \textit{transformers}, voltada para o processamento de imagens. Nesse caso, a composição de \textit{feature vectors} é definida dividindo a imagem examinada em \textit{patches}. Existem diversas formas de realizar essa divisão, porém a mais usual é o formato 16x16 pixels. Após os \textit{patches} serem devidamente distribuídos, uma transformação linear é aplicada, semelhantemente ao BERT, gerando assim os respectivos \textit{embeddings} de posições. Em seguida, os embeddings são analisados por meio de várias camadas, formando uma estrutura de \textit{multi-head attention}. Com isso, cada \textit{patch} de imagem é comparado aos outros, de diferentes formas, atribuindo, portanto, múltiplos valores aos \textit{feature vectors}, tornando o processo de classificação mais preciso \cite{dosovitskiy2020}.
%
%No contexto de aplicação desta proposta, as amostras de dados (ataque ou quebra de segurança de dados) encontram-se no formato de texto. Por isso, é necessário realizar a conversão dos textos para imagem para viabilizar o uso de ViT. Neste trabalho, o modelo que será investigado foi nomeado como \textit{Butterfly Vision Transformer}(B\_ViT) \cite{belal2023global}. É válido ressaltar que o B\_ViT utiliza uma arquitetura ViT pré-treinada, e o que o difere dos demais é o seu mecanismo de \textit{attention}, que também foi modificado. Além de utilizar como entrada códigos executáveis convertidos em imagens em níveis de cinza, B\_ViT opera com dois tipos de \textit{attention}, sendo uma delas a \textit{local attention}, e a outra \textit{global attention}. \textit{local attention} será aplicada em pequenas partes da imagem, os patches, com a intenção de detectar detalhes em uma área restrita, resultando em melhor exploração da imagem dada. Isso permite quantificar informações maliciosas ocultas ou mascaradas. Já a parte de \textit{global attention} será aplicada na imagem completa. Ao aplicar \textit{global attention}, será verificado se o modelo consegue entender a relevância e a abrangência de cada patch, relacionando-os a fim de obter uma interpretação mais ampla sobre um malware. As vantagens em utilizar o B\_ViT se baseiam em capturar tanto detalhes finos quanto padrões amplos, resultando em uma análise mais precisa e completa para a classificação de imagens de exemplos de \textit{malwares}.
%
%Para viabilizar esta etapa, é necessário aplicar técnicas para transformação de texto em imagens (\textit{reshaping}), que se baseiam em transformações lineares aplicadas sobre a representação dos textos geradas pelo modelo. Algumas técnicas que serão exploradas neste trabalho visam permitir o uso do B\_ViT para classificar e reconhecer amostras maliciosas presentes em diferentes \textit{datasets}, tais como \textit{sequential reshape}, \textit{recurrence plot}, \textit{gramian angular field} (GAF) \cite{terzi2022gramian}, e imagens em níveis de cinza. As imagens geradas após a aplicação do processo de \textit{reshaping} serão passadas para o modelo B\_ViT como entrada, possibilitando assim, outra perspectiva quanto à classificação de comportamentos intrusivos presentes nos códigos, que antes eram representados e analisados apenas como texto.
%
%%\subsection{Sequential reshape}
%    %\paragraph{}Os \textit{feature vectors} gerados em fases precedentes ao processo de \textit{reshape} devem ser convertidos em matrizes, de modo a representar os três canais de cores, RGB. Estes vetores devem ser reorganizados sequencialmente em matrizes de dimensões 10x10x3. As matrizes geradas pelas distâncias , $\Delta$e e $\Delta$m correspondem respectivamente aos canais de cores vermelho, verde e azul.
%    %\paragraph{}Posteriormente, com o uso de funções de ativação, é possível gerar através destas representações, uma outra imagem que representa as características da imagem dada como entrada, tornando possível para o modelo realizar a respectiva classificação.
%%\subsection{Recurrence plot}
%    %\paragraph{}Outro modo de representar \textit{feature vectors} como imagem é através da projeção eventos de repetidos em espaços de duas ou três dimensões, ou seja, uma série. Ao aplicar a técnica de \textit{Recurrence Plot}, através de um vetor que contenha \textit{N} valores, será gerada uma matriz \textit{NxN}, onde cada elemento é obtivo pelo cálculo da norma de distância Euclidiana. Assim, como as matrizes geradas para cada distância $\Delta$ é bidimensional, a combinação das três distâncias gera como saída uma imagem RGB.
%%\subsection{Grayscale images}
%%    \paragraph{} \textit{Grayscale images} são imagens convertidas em escala cinza. No contexto do projeto, são de suma importância, pois são o tipo de entrada sobre a qual o modelo B\_ViT opera. Os códigos executáveis, após serem convertidos em imagem, passam ainda por uma conversão para tons de cinza, através da aplicação das técnicas de \textit{reshape} \cite{saravanan2010color}. Destacando, assim, partes consideráveis para a classificação da entrada em questão. 
%\section{Etapa 4 - Contexto de Aplicação: datasets}
%Todos os tipos de dados utilizados no processo de treinamento e aprendizado de um modelo encontram-se armazenados e organizados em um dataset, separados por classes. No contexto do projeto apresentado, os datasets utilizados pelos modelos pré-treinados escolhidos estão concentrados na área da \textit{CyberSegurança}, tais como amostras de ataques dos \textit{distributed denial of service} (DDoS), \textit{man in the middle} (MitM), SQL injection e \textit{cross site scripting} (XSS) \cite{luxemburk2021detection}. Logos, a ideia é verificar a viabilidade das associações previstas neste trabalho nos seguintes datasets: CSIC 2010 \cite{gimenez2010http}, FWAF \cite{ahmad2017web} e \textit{httpParams} \cite{morzeux2020httpparams}. Ao implementar a aplicação BERT, após o registro dos resultados obtidos, os dados de texto que representam os variados ataques cibernéticos passarão pela conversão de texto para imagem. Logo, as técnicas de \textit{reshaping} serão aplicadas em todas as amostras contidas nos datasets, gerando, portanto, um novo banco de imagens, que servirá como base de estudos para o modelo B\_ViT em questão.
%
%\subsection{Etapa 5 - Análises e Extração de Conhecimento}
%
%O processo de análise e avaliação de desempenho ocorrerá por meio de métricas comumente exploradas na área de aprendizado de máquina, tais como acurácia, \textit{medida-F}, \textit{recall} e outras \cite{duda2012pattern}. A validação da melhor combinação ocorrerá por meio de comparações dos desempenhos com: fornecidos por trabalhos correlatos; e, obtidas via as diferentes arquiteturas indicadas nesta proposta.
