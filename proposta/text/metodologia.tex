\section{Metodologia}

A pesquisa será conduzida em cinco etapas principais, dispostas a seguir.


\subsection{Etapa 1 - Revisão Bibliográfica}

Esta etapa permite o levantamento bibliográfico necessário para manter o projeto atualizado, proporcionar uma fundamentação teórica sólida e subsidiar a exploração proposta.


\subsection{Etapa 2 - processar \textit{dataset} de imanges histológicas para aceptabilidade para o padrão do \textit{NeRF's}}

  Dado um conjunto de imagens fornecido pelo % TODO adicionar referencia ao dataset
  , observa-se certa dissonância entre a entrada esperada pelo \textit{pipeline} do campo de radiação neural e as imagens em si. É necessário % TODO

\subsection{Etapa 3 - geração de imanges com GAN's}

ok

\subsection{Etapa 4 - Treinamento do modelo nerf}

ok 

\subsection{Etapa 5 - Validação dos resultados}

Para obter-se resultados precisos, factíveis e replicáveis, este projeto buscará utilizar-se de métricas de validação  tanto para imagens 2D quanto para modelos 3D reconstruídos.

\begin{itemize}

  \item \textbf{Validação de modelos 2D}: Serão empregados, no campo das imagens bidimensionais, tanto o PSNR (\textit{Peak Signal-To-Noise Ratio}) quanto o SSIM (\textit{Structural Similarity Index}). O PSNR faz uma comparação direta entre 2 imagens e consegue descrever o ruído discrepante entre estas, utilizando-se de uma razão sinal-ruído em escala logarítmica. Já o SSIM trabalha com a estrutura da imagem em si, ou seja, nos agrupamentos dos pixels da referência quando comparados à imagem gerada pela arquitetura GAN, capturando melhor a percepção humana de qualidade visual do que o PSNR.

  \item \textbf{Validação de modelos 3D}: Para o domínio das reconstruções 3D, utilizar-se-á o Volumetric Intersection over Union (IoU Volumétrico). Esta métrica calcula a razão entre o volume da interseção e o volume da união do modelo reconstruído e do modelo de referência, fornecendo uma medida direta da sobreposição espacial dos volumes.

Essas métricas fornecerão uma base quantitativa robusta para avaliar e comparar a qualidade dos modelos testados, dimensionando uma alta fidelidade ao obter-se valores elevados de PSNR e SSIM e de IoU Volumétrico.

\end{itemize}
