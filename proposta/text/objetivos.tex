\section{Objetivos}

\begin{itemize}
  \item entender a possibilidade de se utilizar metodos de \textit{data augmentation} para influenciar o resultado da geração de objetos 3D com nerfs.
  \item investigar a viabilidade de se aumentar um dataset histológico ou nao para o uso em nerf.
\end{itemize}


%%% -- 

%Neste projeto, a meta é investigar e aplicar modelos fundamentados em \textit{transformers} para classificar e reconhecer padrões na área de \textit{CyberAtaques}, tais como via requisições HTTP/HTTPS. Para tanto, pretende-se:
%
%\begin{itemize}
%    \item 
%    Explorar um modelo BERT com uma abordagem \textit{multilayer perceptron} (MLP) para classificar e reconhecer os padrões sob investigação;
%    \item 
%    Investigar técnicas para transformação de texto em imagens (\textit{reshaping}) a fim de testar a viabilidade ViT neste tipo aplicação (\textit{CyberAtaques}), tais como amostras de \textit{distributed denial of service} (DDoS), \textit{man in the middle} (MitM), SQL injection ou \textit{cross site scripting} (XSS) \cite{luxemburk2021detection};
%    \item 
%    Explorar o modelo ViT nomeado como \textit{Butterfly Vision Transformer} para classificar as representações obtidas na etapa anterior;
%    \item
%    Analisar as capacidades discriminativas das estratégias em comparação com os modelos disponíveis na literatura especializada;
%    \item Definir as principais associações e limites observados no contexto aqui explorado.
%    
%\end{itemize}
%
%Considerando o previsto em Edital PROPe Unesp Nº 08/2024 - PIBIC, esta proposta tem aderência plena com as áreas Prioritárias do Ministério da Ciência, Tecnologia, Inovações e Comunicações (estabelecidas na Portaria MCTIC nº 1.122/2020, com texto alterado pela Portaria MCTIC nº 1.329/2020), especificamente com as Áreas Tecnologias Estratégicas (Cibernética) e Tecnologias Habilitadoras (Inteligência Artificial). No que tange a lista Objetivos do Desenvolvimento Sustentável (ODS), essa pesquisa apresenta aderência direta com a ODS 9 (indústria, inovação e infraestrutura).
