\section{Resultados Esperados}

Este projeto visa investigar a aplicabilidade de métodos convencionais de Machine Learning (ML) em dados histológicos.  O objetivo principal é explorar técnicas como Data Augmentation, GANs (Generative Adversarial Networks) e NeRFs (Neural Radiance Fields) para superar desafios como a escassez de dados e a complexidade estrutural de tecidos biológicos.
%Espera-se que a execução deste projeto forneça resultados relevantes, tais como: viabilidade do uso de um modelo ViT para classificar e reconhecer padrões na área de \textit{CyberAtaques},  tais como por meio de requisições HTTP/HTTPS; conhecimento sobre as principais associações (\textit{reshaping} com ViT) e limites correspondentes no contexto aqui explorado; indicação da capacidade discriminativa mais relevante entre BERT e B\_ViT para os conjuntos de dados considerados nos experimentos práticos destes modelos, contribuindo para a compreensão e avanço no campo de análise de técnicas direcionadas para \textit{CyberAtaques}.
