\section{Introdução}
\label{sec:introducao}

Em dissonância a métodos tradicionais de síntese de imagens, os campos de radiação neural, os \textit{NeRF's} \cite{mildenhall2020nerfrepresentingscenesneural}, utilizam-se de técnicas de renderização volumétricas comuns, apoiando-se nas redes neurais artificiais MLP's (multilayer perceptron) \cite{GARDNER19982627} para processar a entrada de coordanadas 5D (x, y, z, $\theta$, $\phi$) em pontos de densidade ($\sigma$) e cor ($c = (r, g, b)$) \cite{mildenhall2020nerfrepresentingscenesneural}. Porém, para conseguir modelos tri-dimensionais de alta confiabilidade, o \textit{NeRF} depende de um conjunto consideravelmente grande de imagens e diferentes rotações dos ângulos da câmera do objeto/ambiente de referência.

Considerando-se a redundância supracitada, vários autores buscaram aprimorar seus datasets com inúmeras técnicas. Por exemplo, pesquisadores utilizaram-se do Structure from Motion (SFM), uma técnica de imagem de alcance fotogramétrico, para se capturar imagens do ambiente em questão com maior precisão e qualidade, como no método de gaussianos 3D \cite{kerbl3Dgaussians}, ou mesmo o MVS (multi-view stereo) \cite{garbin2021fastnerfhighfidelityneuralrendering}. Estas técnicas, conquanto fiéis à sua referência, ou são computacionalmente caras ou, ainda, precisam de uma quantidade exagerada imagens do item a ser renderizado. \textit{Datasets} histológicos, no entanto, não compreender de um grupo de exames de imagens tão robusto, com a redundância de ângulos e inforações necessários. A incerteza acerca destas imagens, a escassez de dados e a complexidade estrutural dos tecidos biológicos são desafios significativos na aplicação de técnicas de aprendizado profundo em imagens histológicas.  

Portanto, neste trabalho, buscar-se-á possibilidades de se minimizar a distância entre um \textit{dataset} ideal (repleto de redundância e ângulos de câmera distintos). Para tal, seguindo trabalhos similares, se faz necessária uma investigação sobre a viabilidade de se utilizar técnicas de geração de imagem como GAN's (\textit{generative adversarie network}) \cite{goodfellow2014generativeadversarialnetworks} aplicadas em dados histológicos, como feito previamente por Rozendo \cite{rozendo2024histdataaug} em imagens bidimensionais.


\newpage
