%\singlespacing
\section{Introdução}
\label{sec:introducao}

Diferentemente de métodos tradicionais de renderização, os campos de radiação neural, os \textit{NeRF's} \cite{mildenhall2020nerfrepresentingscenesneural} utilizam-se de técnicas de renderização volumétricas comuns, se apoiando nas redes neurais artificiais MLP's (multilayer perceptron) \cite{GARDNER19982627} para processar a entrada de coordanadas 5D em pontos de densidade e cor \cite{mildenhall2020nerfrepresentingscenesneural}, em contraponto a outros modelos que se baseiam em voxels ou malhas comuns nas técnicas de renderização padrão. Para isso, o \textit{NeRF} precisa de um conjunto consideravelmente grande de imagens e diferentes rotações de ângulos.

Considerando-se a problemática supracitada, vários autores buscam (ainda) aprimorar seus datasets com inúmeras técnicas, por exemplo, utilizando-se do SFM - Structure from Motion%\cite{}
, uma técnica de imagem de alcance fotogramétrico, para se capturar imagens do ambiente em questão com maior precisão e qualidade.

alcançando técnicas computacionalmente caras ou que, ainda assim, precisam de diversas imagens do ambiente/objeto a ser renderizado\cite{kerbl3Dgaussians}.

Entretanto, \textit{datasets} histológicos tendem a depender de grupo de exames de imagens os quais nem sempre possuem a redundância de ângulos necessários para se produzir um bom objeto 3D com NeRF (\textit{Neural Radiation Field})\cite{mildenhall2020nerfrepresentingscenesneural}. Isto posto, a incerteza acerca destas imagens torna o processamento de modelos de NeRF sempre mais complexos.
A escassez de dados e a complexidade estrutural dos tecidos biológicos são desafios significativos na aplicação de técnicas de aprendizado profundo em imagens histológicas.  

Portanto, o objetivo deste trabalho é