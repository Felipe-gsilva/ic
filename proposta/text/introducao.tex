\section{Introdução}
\label{sec:introducao}

Em dissonância a métodos tradicionais de síntese de imagens, os campos de radiação neural, os \textit{NeRF's} \cite{mildenhall2020nerfrepresentingscenesneural}, utilizam-se de técnicas de renderização volumétricas comuns, apoiando-se nas redes neurais artificiais MLP's (multilayer perceptron) \cite{GARDNER19982627} para processar a entrada de coordanadas 5D em pontos de densidade e cor \cite{mildenhall2020nerfrepresentingscenesneural}. Porém, para conseguir modelos tri-dimensionais de alta confiabilidade, o \textit{NeRF} depende de um conjunto consideravelmente grande de imagens e diferentes rotações dos ângulos da câmera do item de referência.

Considerando-se a redundância supracitada, vários autores buscaram aprimorar seus datasets com inúmeras técnicas. Por exemplo, utilizando-se do Structure from Motion (SFM), uma técnica de imagem de alcance fotogramétrico, para se capturar imagens do ambiente em questão com maior precisão e qualidade, alcançando técnicas computacionalmente caras ou que, ainda assim, precisam de diversas imagens do ambiente/objeto a ser renderizado\cite{kerbl3Dgaussians}.

Entretanto, \textit{datasets} histológicos tendem a depender de grupo de exames de imagens os quais nem sempre possuem a redundância de ângulos necessários. A incerteza acerca destas imagens, a escassez de dados e a complexidade estrutural dos tecidos biológicos são desafios significativos na aplicação de técnicas de aprendizado profundo em imagens histológicas.  

Portanto, o objetivo deste trabalho é