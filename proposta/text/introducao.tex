%\singlespacing
\section{Introdução}
   Com a evolução da tecnologia, o mundo atual encontra-se significativamente conectado via diversos dispositivos eletrônicos integrados à internet. O fluxo de informação que circula pela internet é extremamente volumoso, podendo conter qualquer tipo de dado, sendo este legítimo ou malicioso. Logo, "software maliciosos" (malware), códigos de software ou programa de computadores intencionalmente escrito para prejudicar um sistema de computador ou seus usuários, são constantemente aprimorados para difentes crimes cibernéticos, tais como: utilizar dispositivos, dados ou redes corporativas inteiras reféns para obter recursos financeiros; ter acesso não autorizado de dados confidenciais ou ativos digitais; obter credenciais de sistemas, números de cartão de crédito ou propriedades intelectuais; interromper sistemas críticos de empresas e/ou agências governamentais. A comunicação pela internet é estabelecida via pedidos e respostas, que seguem um protocolo, sendo o \textit{hypertext transfer protocol} (HTTP) um dos mais comuns. Servidores HTTP estão abertos para qualquer tipo de requisição, que podem conter informações maliciosas e que caracterizam distintas ameaças. Também, é notável o crescente número de ataques cibernéticos registrados pela internet, com rápida propagação e visível sofisticação, seja por e-mails, links contaminados, aplicativos maliciosos ou \textit{malwares}. 
   
   Métodos desenvolvidos para detectar malwares são extremamente necessários, com destaque para soluções fundamentadas em técnicas de inteligência artificial. Entretanto, como abordado no estudo de Xuejun Zong et al. \cite{he2024network}, modelos obtidos via CNNs, RNNs e GANs possuem limitações que representam obstáculos significativos para uma aplicação prática efetiva, impactando o uso e a confiabilidade destes em contextos críticos. Por outro lado, o uso e abordagens \textit{transformers} têm minimizado parte das dificuldades, com modelos capazes de indicar as maiores taxas de sucesso e com os melhores níveis de confiabilidade. Utilizando os \textit{encoders}, componentes responsáveis por transformar as entradas (imagem ou texto) em números, que podem assim finalmente ser interpretados pelo modelo, os \textit{transformers} se diferenciam das outras arquiteturas devido ao mecanismo de \textit{attention}. Este consiste em realizar comparações com o conteúdo dos \textit{datasets} utilizados no treinamento e examinar adicionalmente cada \textit{token} relacionado aos outros \textit{tokens} de entrada, provendo diversos valores semânticos para um mesmo dado. Este fato possibilita ao modelo compreender o contexto no qual um determinado \textit{token} está inserido. 
   
   Neste contexto, o \textit{Bidirectional Encoder Representations from Transformers} (BERT) e os \textit{Vision Transformers} (ViT) merecem destaque em tarefas de classificação e reconhecimento de padrões de textos e imagens, respectivamente. Assim, considerando os avanços indicados previamente, é possível investigar o poder discriminativos dessas técnicas na área da \textit{cybersegurança}, especificamente no campo de identificação de $malware$. BERT tem sido amplamente utilizado para o reconhecimento e classificação de tipos de \textit{malware}, tais como registros de captura de pacotes do tráfego de rede, \textit{SecurityBERT} \cite{ferrag2024revolutionizing}, domínios maliciosos, código binário executável de programas, e até mesmo logs dos processos registrados por sistemas em dispositivos, como o CyBERT \cite{ranade2021cybert}. Por outro lado, modelo ViT é uma tecnologia ainda pouca explorada neste tipo de aplicação, mesmo com algumas abordagens relevantes na área de aplicação desta proposta, como B\_ViT \cite{belal2023global} e DE-ViT \cite{he2024network}. A aplicação de ViT neste contexto torna-se viável a partir da representação de texto em imagens. Portanto, esse tipo de estudo e investigação permite contribuições relevantes para a classificação e o reconhecimento de padrões na área de \textit{cybersegurança}, por fornecerem testes ainda não explorados, especialmente via diferentes técnicas para representação de textos (dados que podem caracterizar cada tipo de $$malware$$) comumente  em imagens. Os conhecimentos obtidos podem embasar o desenvolvimento de sistemas computacionais mais completos e confiáveis, além de observar possíveis limites no contexto de ViT para essa finalidade.
