\section{Introdução}
\label{sec:introducao}

A histologia é uma área da biologia que estuda os tecidos biológicos em nível microscópico, sendo fundamental para a compreensão de diversas condições médicas e o desenvolvimento de possíveis tratamentos, diagnósticos e terapias. Tradicionalmente, a análise histológica envolve a coleta de amostras de tecidos, sua preparação e coloração, seguida pela observação microscópica, tradicionalmente computadas em cortes bidimensionais (2D). No entanto, com o avanço das tecnologias informacionais, novas abordagens estão sendo desenvolvidas para melhorar a eficiência das avaliações e inferências sobre os tecidos. Embora cruciais para diagnóstico e pesquisa, as imagens 2D possuem uma limitação para a visualização de arquiteturas e tecidos complexos. Desta forma, a renderização 3D se introduz nesta seara ao facilitar a visualização integrada de tais, como órgãos, vasos sanguíneos ou tecidos orgânicos com perspectivas dinâmicas, aprimorando a compreensão de relações espaciais críticas para diagnósticos e planejamentos cirúrgicos. Além disso, modelos 3D personalizados, gerados a partir de dados de pacientes, podem simular intervenções médicas com maior segurança, servir como ferramenta educativa para treinamento de profissionais e até apoiar a criação de próteses ou implantes sob medida. Existem diversas metodologias e técnicas para se alcançar bons resultados e modelos 3D, seja através da modelagem a partir de um artista ou, de maneira avançada, obter-se modelos a partir das já existentes imagens.

Em dissonância a métodos tradicionais de síntese gráfica, os campos de radiação neural (NeRF) utilizam-se de técnicas de renderização volumétricas comuns, apoiando-se nas redes neurais artificiais MLP's (multilayer perceptron) para processar a entrada de coordanadas 5D ($(x, y, z, \theta, \phi)$, cada posição compreendida como um vetor 3D de posições $(x, y, z)$ computado com o ângulo e direção de câmera correspondentes $(\theta, \phi)$) em pontos de densidade volumétrica ($\sigma (x)$) e cor ($c = (r, g, b)$), capazes de, posteriormente, se transformarem em volume 3D convencional \cite{mildenhall2020nerfrepresentingscenesneural}. Porém, para conseguir modelos de alta confiabilidade, o \textit{NeRF} depende de um conjunto consideravelmente grande de imagens e diferentes rotações dos ângulos da câmera do objeto de referência. Vários autores buscaram (e buscam) aprimorar seus datasets com inúmeras técnicas, sempre buscando qualificar mais e mais os dados e, consequentemente, obter imagens mais próximas do \textit{ground-truth}. Por exemplo, pesquisadores utilizaram-se do SFM  (Structure from Motion), uma técnica de imagem de alcance fotogramétrico, para se capturar imagens do ambiente em questão com maior precisão e qualidade, como no método de gaussianos 3D \cite{kerbl3Dgaussians}, ou mesmo o MVS (multi-view stereo), escolhido pelo uso subsequente de CNN's (\textit{convulutional neural network}) \cite{chen2021mvsnerffastgeneralizableradiance}, tudo isso chegando no estado da arte. Estas técnicas, conquanto fiéis às suas referências, são, ora computacionalmente caras, ora, ainda, demandantes de uma quantidade exacerbada de imagens do item a ser renderizado. \textit{Datasets} histológicos, entretanto, costumam não possuir tamanha diversidade ou redundância de ângulos necessária, devido aos desafios como a complexidade estrutural dos tecidos e restrições éticas na coleta de amostras \cite{XUE2021101816}.

Consonantemente, neste trabalho, buscar-se-á possibilidades de se minimizar a distância entre um \textit{dataset} ideal (repleto de redundância e ângulos de câmera distintos) e os datasets comuns na histologia. Para tal, seguindo trabalhos similares, se faz necessária uma investigação sobre a viabilidade de se utilizar técnicas de Deep Learing, seja com pré-processamento intensivo das imanges, ou mesmo com a geração de com GAN's (\textit{generative adversarie network}), como feito previamente por Rozendo \cite{rozendo2024histdataaug} em imagens bidimensionais. GAN's  permitem a criação de imagens através de duas redes neurais treinadas para competição mútua (G - gerador e  D - discriminador). G tenta criar imagens que emulam o pertencimento ao domínio de D, enquanto este o avalia, classificando-o como pertencente ou não. Eventualmente, G e D atingem um estado de equilíbrio (chamado de equilibrio de Nesh), no qual, mesmo após maiores treinamentos, a diferença não mais se altera significativamente \cite{goodfellow2014generativeadversarialnetworks} . Ao utilzizar-se destes conceitos, espera-se, portanto, promover a redução da dependência de \textit{datasets} extensos de \textit{NeRF}'s, enquanto mantem-se sua qualidade instrínseca.
